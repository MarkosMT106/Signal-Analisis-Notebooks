\documentclass{article}
\usepackage[spanish]{babel}
\usepackage[utf8]{inputenc}
\usepackage{times}
\usepackage{amsmath}
\usepackage{amsfonts}
\usepackage{amssymb}
\usepackage{authblk}
\usepackage{ulem}
\usepackage{graphicx}
\usepackage{natbib}
\usepackage{mathrsfs}
\usepackage{amsthm}

\renewcommand{\Re}{\operatorname{\rm Re}}
\renewcommand{\Im}{\operatorname{\rm Im}}

\title{Serie de Fourier}
\author[1]{Josué Tago Pacheco\thanks{josue.tago@gmail.com}\\
Facultad de Ingeniería\\
Universidad Nacional Autónoma de México}

%\renewcommand\Authands{ and }
%\setcounter{secnumdepth}{5}
%\numberwithin{equation}{section}

\spanishdecimal{.}

\begin{document}
\maketitle

\begin{abstract}
Estas notas fueron creadas para sustituir la sección 1.4 y la primera parte de la sección 1.4.1 de las Notas de \citet{Anguiano_1996_IAF} que se llevan en el curso de An\'alisis Espectral de Señales en el 6to. Semestre de la Carrera de Ingenier\'ia Geof\'isica impartida por la Facultad de Ingenier\'ia de la UNAM. 
\end{abstract}

\section*{Aproximaci\'on de funciones por funciones ortogonales complejas}
Dada una funci\'on $f(t)$, \'esta puede aproximarse mediante funciones complejas que sean ortogonales de manera que
\begin{equation}
f_a(t)=\tilde{\mathscr{D}}_0\mathscr{G}_0(t)+[\tilde{\mathscr{D}}_0\mathscr{G}_0(t)]^*+\tilde{\mathscr{D}}_1\mathscr{G}_1+[\tilde{\mathscr{D}}_1 \mathscr{G}_1]^*+\cdots+\tilde{\mathscr{D}}_n\mathscr{G}_n(t)+[\tilde{\mathscr{D}}_n\mathscr{G}_n(t)]^*, \label{eq:fa}
\end{equation} 
donde $\tilde{\mathscr{D}}_k$ representa a un coeficiente complejo, $\mathscr{G}_k$ a una funci\'on compleja y $^*$ simboliza el conjugado de la funci\'on compleja. Como nuestra funci\'on original es real, se suma el conjugado de la funci\'on compleja para que la parte imaginaria sea cero, aunque se obtenga dos veces la parte real.
\subsection*{Serie Finita de Fourier en su forma compleja}
Un conjunto de funciones complejas que cumplen con la condici\'on de ortogonalidad son las funciones exponenciales complejas definidas como
\begin{equation}
\mathscr{G}_k(t)=\exp^{\frac{i2\pi kt}{T}}, \label{eq:gk}
\end{equation}
donde $T$ se es el \textit{periodo}. El conjugado de la ec. (\ref{eq:gk}) est\'a dado por
\begin{equation}
\mathscr{G}_k^*(t)=\exp^{-\frac{i2\pi kt}{T}}.
\end{equation}
La condición de ortogonalidad es
\begin{equation}
\int_{t_1}^{t_1+T}\exp^{\frac{i2\pi kt}{T}}\exp^{-\frac{i2\pi jt}{T}}dt=\left\{\begin{array}{cc}
0 & \textrm{si }k\neq j \\
T & \textrm{si }k=j.
\end{array}\right.
\end{equation}
\begin{proof}
La demostraci\'on se har\'a para cada uno de los posibles casos
\begin{itemize}
\item Caso $k\neq j$
\begin{eqnarray}
\int_{t_1}^{t_1+T}\exp^{\frac{i2\pi kt}{T}}\exp^{-\frac{i2\pi jt}{T}}dt&=&\int_{t_1}^{t_1+T}\exp^{\frac{i2\pi(k-j)t}{T}}dt \nonumber \\
&& \textrm{haciendo $l=k-j$ y utilizando coordenadas polares} \nonumber \\
&=& \int_{t_1}^{t_1+T}\cos\left(\frac{2\pi lt}{T}\right)+i\sin\left(\frac{2\pi lt}{T}\right)dt. \label{eq:po_1}
\end{eqnarray}
Trabajando con el primer t\'ermino de la integral (\ref{eq:po_1})
\begin{eqnarray}
\int_{t_1}^{t_1+T}\cos\left(\frac{2\pi lt}{T}\right)dt&=&\left(\frac{T}{2\pi l}\right)\sin\left.\left(\frac{2\pi lt}{T}\right)\right|_{t_1}^{t_1+T} \nonumber \\
&=&\left(\frac{T}{2\pi l}\right)\left(\sin\left(\frac{2\pi l t_1}{T}+2\pi l\right)-\sin\left(\frac{2\pi lt_1}{T}\right)\right) \nonumber \\
&&\textrm{Usando la identidad trigonom\'etrica de }\sin(\theta_1+\theta_2) \nonumber \\
&=&\left(\frac{T}{2\pi l}\right)\left(\sin\left(\frac{2\pi l t_1}{T}\right)\cos(2\pi l)+\cos\left(\frac{2\pi lt_1}{T}\right)\sin(2\pi l)-\sin\left(\frac{2\pi lt_1}{T}\right)\right) \nonumber \\
&=&\left(\frac{T}{2\pi l}\right)\left(\sin\left(\frac{2\pi lt_1}{T}\right)\left[-1+\cos(2\pi l)\right]+\cos\left(\frac{2\pi lt_1}{T}\right)\sin(2\pi l)\right) \nonumber \\
&&\textrm{Como }\cos(2\pi l)=1 \textrm{ y }\sin(2\pi l)=0 \nonumber \\
&=&0. \nonumber
\end{eqnarray}
Trabajando con el segundo término de la integral (\ref{eq:po_1})
\begin{eqnarray}
\int_{t_1}^{t_1+T}\sin\left(\frac{2\pi l}{T}t\right)dt&=&\left(\frac{T}{2\pi l}\right)\left(-\left.\cos\left(\frac{2\pi l}{T}t\right)\right|_{t_1}^{t_1+T}\right) \nonumber \\
&=&\left(\frac{T}{2\pi l}\right)\left(-\cos\left(\frac{2\pi lt_1}{T}+2\pi l\right)+\cos\left(\frac{2\pi lt_1}{T}\right)\right) \nonumber \\
&&\textrm{Usando la identidad trigonom\'etrica de }\cos(\theta_1+\theta_2) \nonumber \\
&=&\left(\frac{T}{2\pi l}\right)\left(-\left[\cos\left(\frac{2\pi lt_1}{T}\right)\cos(2\pi l)-\sin\left(\frac{2\pi lt_1}{T}\right)\sen(2\pi l)\right]+\cos\left(\frac{2\pi lt_1}{T}\right)\right) \nonumber \\
&=&\left(\frac{T}{2\pi l}\right)\left(\cos\left(\frac{2\pi l t_1}{T}\right)[1-\cos(2\pi l)]-\sin\left(\frac{2\pi lt_1}{T}\right)\sin(2\pi l)\right) \nonumber \\
&=&0. \nonumber
\end{eqnarray}
\item Caso $k=j$ 
\begin{eqnarray}
\int_{t_1}^{t_1+T}\exp^{\frac{i2\pi jt}{T}}\exp^{\frac{-i2\pi jt}{T}}dt&=&\int_{t_1}^{t_1+T}dt \nonumber \\
&=&T. \nonumber 
\end{eqnarray}
\end{itemize}
\end{proof}
Sustituyendo las funciones exponenciales complejas en $f_a(t)$ (\ref{eq:fa}) se tiene
\begin{eqnarray}
f_a(t)&=&\tilde{\mathscr{D}}_0\exp^{\frac{i2\pi 0t}{T}}+\tilde{\mathscr{D}}_0^*\exp^{-\frac{i2\pi 0t}{T}}+\tilde{\mathscr{D}}_1\exp^{\frac{i2\pi 1t}{T}}+\tilde{\mathscr{D}}_1^* \exp^{-\frac{i2\pi 1t}{T}}+\cdots+ \nonumber \\
&&\tilde{\mathscr{D}}_n\exp^{\frac{i2\pi nt}{T}}+\tilde{\mathscr{D}}_n^*\exp^{-\frac{i2\pi nt}{T}} \nonumber \\
&=&[\tilde{\mathscr{D}}_0+\tilde{\mathscr{D}}_0^*]+\tilde{\mathscr{D}}_1\exp^{\frac{i2\pi t}{T}}+\tilde{\mathscr{D}}_1^* \exp^{-\frac{i2\pi t}{T}}+\cdots+ \nonumber \\
&&\tilde{\mathscr{D}}_n\exp^{\frac{i2\pi nt}{T}}+\tilde{\mathscr{D}}_n^*\exp^{-\frac{i2\pi nt}{T}} \nonumber \\
&=&2\Re[\tilde{\mathscr{D}}_0]+\tilde{\mathscr{D}}_1\exp^{\frac{i2\pi t}{T}}+\tilde{\mathscr{D}}_1^* \exp^{-\frac{i2\pi t}{T}}+\cdots+ \nonumber \\
&&\tilde{\mathscr{D}}_n\exp^{\frac{i2\pi nt}{T}}+\tilde{\mathscr{D}}_n^*\exp^{-\frac{i2\pi nt}{T}}. \label{eq:fa_exp}
\end{eqnarray}
Si de manera conveniente hacemos que $\mathscr{D}_{k}=\tilde{\mathscr{D}}_k$, $\mathscr{D}_{-k}=\tilde{\mathscr{D}}_k^*$ y $\mathscr{D}_0=2\Re[\tilde{\mathscr{D}}_0]$, podemos cambiar los l\'imites de la suma (\ref{eq:fa_exp}) de $-N$ a $N$ como
\begin{equation}
f_a(t)=\sum_{k=-N}^N\mathscr{D}_k\exp^{\frac{i2\pi kt}{T}},
\end{equation}
la cual se conoce como \textbf{serie de Fourier}.\\
Para calcular sus coeficientes, $\mathscr{D}_k$, seguimos el m\'etodo del error cuadr\'atico medio m\'inimo (ECMM).\\
El error cuadr\'atico medio est\'a dado por
\begin{eqnarray}
E^2(t)&=&\frac{1}{T}\int_{t_1}^{t_1+T}(f(t)-f_a(t))^2 dt \nonumber \\
&=&\frac{1}{T}\int_{t_1}^{t_1+T}\left(f(t)-\sum_{k=-N}^N\mathscr{D}_k\exp^{\frac{i2\pi kt}{T}}\right)^2 dt. \label{eq:ec}
\end{eqnarray}
Los coeficientes que minimizan al error cuadr\'atico (\ref{eq:ec}) cumple con
\begin{equation}
\frac{\partial E^2}{\partial\mathscr{D}_k}=0\qquad\forall k.
\end{equation}
Considerando la derivada con respecto al $j$-\'esimo coeficiente
\begin{eqnarray}
\frac{\partial E^2(t)}{\partial\mathscr{D}_j}&=&\frac{\partial}{\partial\mathscr{D}_j}\left[\frac{1}{T}\int_{t_1}^{t_1+T}\left(f(t)-\sum_{k=-N}^N\mathscr{D}_k\exp^{\frac{i2\pi kt}{T}}\right)^2 dt\right] \nonumber \\
&=&\frac{1}{T}\int_{t_1}^{t_1+T}2\left(f(t)-\sum_{k=-N}^N\mathscr{D}_k\exp^{\frac{i2\pi kt}{T}}\right)\left( -\exp^{\frac{i2\pi jt}{T}}\right)dt \nonumber \\
&=&\textrm{Utilizando la condici\'on de ortogonalidad , s\'olo sobrevive el t\'ermino }k=-j \nonumber \\
&=&\frac{2}{T}\left(\int_{t_1}^{t_1+T}\left(-f(t)\exp^{\frac{i2\pi jt}{T}}\right)dt+\mathscr{D}_{-j}T\right) \nonumber \\
&\Rightarrow &\mathscr{D}_{-j}=\frac{1}{T}\int_{t_1}^{t_1+T}f(t)\exp^{\frac{i2\pi jt}{T}}dt,
\end{eqnarray}
que podemos escribir como
\begin{equation}
\mathscr{D}_k=\frac{1}{T}\int_{t_1}^{t_1+T}f(t)\exp^{-\frac{i2\pi kt}{T}}dt,
\end{equation}
usando el cambio de variable $k=-j$.

% Bibliography
\bibliographystyle{apalike}
\bibliography{sec_1p4.bib}
\end{document}
%\begin{figure}[h] 
% \centering
%   \includegraphics[width = 1.0 \textwidth]{Figures/phi_function_shape.eps}      
%   \caption{Show the shape of the phi function} \label{fig:phi}
%\end{figure}